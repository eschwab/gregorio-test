\documentclass[a5paper,11pt,DIV=17]{scrartcl}

\usepackage[autocompile,allowdeprecated=false]{gregoriotex}
\usepackage{libertine}
\usepackage{calc}
\pagestyle{empty}
\grechangestaffsize{13}
\newlength\tmpsize
\makeatletter
\setlength\tmpsize{\gre@factor pt}
\makeatother
\grechangestyle{initial}{\fontsize{2.5\tmpsize}{2.5\tmpsize}\selectfont}

\begin{document}

First, let's show default behaviour :

\greannotation{\scriptsize Test}
\greannotation{\scriptsize 1.}
\gabcsnippet{(f3) Dó(f)mi(fgf___)ne.(f.,ef!hhkxi./jkI'Gh_ihhf.) (::)}

\smallskip
Here, one would expect second annotation to "touch" the first :

\grechangedim{annotationseparation}{0mm}{0}
\greannotation{\scriptsize Test}
\greannotation{\scriptsize 1.}
\gabcsnippet{(f3) Dó(f)mi(fgf___)ne.(f.,ef!hhkxi./jkI'Gh_ihhf.) (::)}

\smallskip
So, let's try a negative value : one would expect the first not to move, but the second to raise ;
the space between annotations stays the same, and both raise.

\grechangedim{annotationseparation}{-2mm}{0}
\greannotation{\scriptsize Test}
\greannotation{\scriptsize 1.}
\gabcsnippet{(f3) Dó(f)mi(fgf___)ne.(f.,ef!hhkxi./jkI'Gh_ihhf.) (::)}

\smallskip
With a positive value, the second annotation is lower, but the first too (which is counter-intuitive) :

\grechangedim{annotationseparation}{3mm}{0}
\greannotation{\scriptsize Test}
\greannotation{\scriptsize 1.}
\gabcsnippet{(f3) Dó(f)mi(fgf___)ne.(f.,ef!hhkxi./jkI'Gh_ihhf.) (::)}

\end{document}
